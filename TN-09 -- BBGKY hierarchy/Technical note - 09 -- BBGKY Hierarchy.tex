Begin by defining a 6-dimensional phase space quantity $X_i = X_i(t)$ pertaining to any of $i\in\mathbb{N}_1^N$ particles' positions in phase space $X_i = (\vec{x}_i, \vec{v}_i)= \sum_{\ell = 1}^3 (x_{\ell} \hat{x}_{\ell} + v_{\ell} \hat{v_x}_{\ell})$. Here, $(\vec{x}_i, \vec{v}_i)\in\mathbb{R}^3\times\mathbb{R}^3$, hence the collection of all $N$ particles constitute a $6N$-dimensional phase space. For further convenience, define a 6N-dimensional differential volume element  $d\Gamma = \prod_{i=1}^N d^6X_i$ at any time $t$. We can also write for brevity at time $t = 0$, $d\Gamma_0 = \prod_{i=1}^N d^6X_i(0)$. Each particle evolves according to the characteristics:

$$\frac{d\vec{x}_i(t)}{dt} = \vec{v}_i (t), \qquad \frac{d\vec{v}_i}{dt} = \frac{q}{m}\vec{E}(t,\vec{x}_i)$$

We can then define the probablity density function $D_N$ for all $N$ particles such that 

$$D_N(X_1, X_2, \ldots , X_N)d\Gamma = D_N(X_1(0), X_2(0), \ldots X_N(0)) d\Gamma_0$$

is constant in time since no particles are destroyed or created. Since no coordinate transformation has occured it is clear that

$$d\Gamma = d\Gamma_0 \qquad \text{symplecticity of the probability space}$$

Thus, 

$$D_N = D_N(0)$$

Differentiating this with time:

\begin{eqnarray*}
\frac{d}{dt}\left[D_N(X_1, X_2, \ldots X_N)\right] & = & \frac{d}{dt}\left[D_N(X_1(0), X_2(0), \ldots X_N(0)\right]\\
\left(\frac{\partial}{\partial t} + \sum_i \frac{\partial X_i}{\partial t}\frac{\partial }{\partial X_i}\right) D_N & = & 0 \\
\end{eqnarray*}

Now,

$$\frac{dX_i}{dt} = \frac{d}{dt}(\vec{x}_i + \vec{v}_i) = \dot{\vec{x}}_i + \dot{\vec{v}}_i = \vec{v}_i + \frac{q}{m}\vec{E}$$

Then, the above becomes

$$\left(\frac{\partial}{\partial t} + \sum_i \left(\vec{v}_i\cdot\frac{\partial}{\partial\vec{x}_i}  + \frac{q}{m}\vec{E}\cdot \frac{\partial}{\partial\vec{v}_i}\right)\frac{d}{dX_i}\right) D_N  =  0$$


This can be put in bracket form by writing 

$$\vec{E} = -\vec{\nabla}_x\phi = -\vec{\nabla}\sum_{j\neq i} \frac{q_j}{|x_i - x_j|}$$

So that Liouville's equation is equivalent to

$$\left(\frac{\partial}{\partial t} + \sum_i \left(\vec{v}_i\cdot\frac{\partial}{\partial\vec{x}_i}  - \frac{q}{m}\vec{\nabla}\phi\cdot \frac{\partial}{\partial\vec{v}_i}\right))D_N = \frac{d}{dt}D_N = 0$$

and

$$H_N = \sum_i \left(\vec{v}_i\cdot\frac{\partial}{\partial\vec{x}_i}  + \frac{q}{m}\vec{E}\cdot \frac{\partial}{\partial\vec{v}_i}\right)$$

So that Liouville's equation is equivalent to

$$\left(\frac{\partial}{\partial t} + H_N)D_N = \frac{d}{dt}D_N = 0$$


















